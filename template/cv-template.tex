\documentclass[italian,a4paper,nologo]{europasscv}

\usepackage[english]{babel}
\usepackage{enumitem}
\usepackage{csquotes}

\newlist{inlinelist}{enumerate*}{1}
\setlist[inlinelist,1]{label=\roman*)}
\setenumerate[1]{itemsep=.5em}

\ecvname{<Complete Name>}
\ecvaddress{<Address>}
\ecvmobile{<Mobile Phone Number>}
\ecvemail{<E-Mail>}
\ecvgithubpage{<GitHub Page>}

% \ecvgender{Female}
\ecvdateofbirth{<Birthday>}
\ecvnationality{Italiano}

% \ecvpicture[width=3.8cm]{picture.jpg}

\begin{document}
  \begin{europasscv}

  \ecvpersonalinfo
  \ecvsection{Formazione}
       \ecvtitle{Sett. 2016--Giu. 2021}{Diploma in Informatica}
       \ecvitem{}{Istituto Tecnico Superiore J.F.Kennedy in Provincia di Pordenone (FVG)}

       \ecvtitlelevel{Sett. 2018--Giu. 2019}{Certificazione Cisco - IT Essentials}{CCNA}
       \ecvitem{}{%
          \begin{ecvitemize}
               \item[\S] Funzionamento e l'installazione dei sistemi operativi (Linux e Windows);
               \item[\S] Configurazione hardware di un PC.
          \end{ecvitemize}
       }

       \ecvtitlelevel{Sett. 2019--Giu. 2020}{Certificazione Cisco - Introduction to Networks}{CCNA}
       \ecvitem{}{%
          \begin{ecvitemize}
               \item[\S] Concetti teorici sul networking;
               \item[\S] Basi del sistema operativo di Cisco;
               \item[\S] Configurazione di una rete LAN;
               \item[\S] Configurazione dei servizi (DHCP, DNS, VPN..);
               \item[\S] Funzionamento delle apparecchiature di rete (Switch L2/L3, Hub, Router);
               \item[\S] Progettazione di una rete a prova di guasti (Router virtuale, Load-Balancing);
               \item[\S] Configurazione hardware di un PC.
          \end{ecvitemize}
       }

       \ecvtitlelevel{Sett. 2019--Giu. 2020}{Certificazione Cisco - Introduction to Cybersecurity}{CCNA}
       \ecvitem{}{%
          \begin{ecvitemize}
               \item[\S] Riconoscimento delle diverse tipologie di minacce per una rete di dispositivi.
          \end{ecvitemize}
       }

       \ecvtitlelevel{Sett. 2019--Giu. 2020}{Certificazione Cisco - Cybersecurity Essentials}{CCNA}
       \ecvitem{}{%
          \begin{ecvitemize}
               \item[\S] Messa in sicurezza di una rete;
               \item[\S] Conoscenza generale sulle apparecchiature di sicurezza come IDS, IPS e Firewalls;
               \item[\S] Configurazione di un firewall su piattaforma Cisco.
          \end{ecvitemize}
       }

       \ecvsection{Competenze Tecniche}
       \ecvitem{\textcolor{ecvsectioncolor}{Conoscenza Approfondita}}{%
            \begin{inlinelist}
            \item Java
            \item C
            \item MySQL
            \item Bash
            \item Lua
            \item MarkDown
            \end{inlinelist}
       }
       \ecvitem{\textcolor{ecvsectioncolor}{Conoscenza di Base}}{%
            \begin{inlinelist}
            \item C++
            \item Golang
            \item JavaScript
            \item C\#
            \item SQL Server
            \item Racket
            \item PHP
            \item LaTeX
            \end{inlinelist}
       }
       \ecvitem{\textcolor{ecvsectioncolor}{Software di aiuto/sviluppo}}{%
            \begin{inlinelist}
            \item Git
            \item Docker
            \item GNU Make
            \end{inlinelist}
       }

  \ecvsection{Lingue}
       \ecvmothertongue{Italiano}
       \ecvlanguageheader
       \ecvlanguage{Inglese}{C1}{B2}{C1}{B2}{B2}
       \ecvlanguagefooter
  
  \ecvblueitem{Patenti}{AM, B}
  
  \nocite{*}
  \renewcommand{\section}[2]{\ecvsection{#2}}

\end{europasscv}
  
\end{document}
